\documentclass[a4paper]{article}

%% Language and font encodings
\usepackage[english]{babel}
\usepackage[utf8x]{inputenc}
\usepackage[T1]{fontenc}
\usepackage{lipsum}
% \renewcommand\Authfont{\fontsize{12}{14.4}\selectfont}

%% Sets page size and margins
\usepackage[a4paper,top=1.5cm,bottom=1.75cm,left=1.5cm,right=1.5cm,marginparwidth=2cm]{geometry}

%% Useful packages
\usepackage{amsmath}
\usepackage{graphicx}
\usepackage[colorinlistoftodos]{todonotes}
\usepackage[colorlinks=true, allcolors=blue]{hyperref}

\title{\textbf{ARM11 Final Report}}
\author{
  Ashvin Arsakularatne\\
  \texttt{aa9220@ic.ac.uk}
  \and
  Kavya Chopra\\
  \texttt{kc2320@ic.ac.uk}
  \and
  Siddhant Singh\\
  \texttt{ss5120@ic.ac.uk}
  \and
  Ye Lun Yang\\
  \texttt{yly19@ic.ac.uk}
}

\begin{document}
\maketitle

\section{Assembler}
\subsection{The structure:}
\subsubsection{First Pass: }
During the first pass, the assembler initialises the linked list and the symbol table using the \verb|init_linked_list()| and \verb|init_symbol_table()| respectively in \verb|assemble.c|. The linked list is initialised by setting the head to \verb|NULL| and allocating memory based on the size of the linked list. The symbol table is initialised by allocating contiguous memory, setting the capacity to \verb|64| and the key-value pairs to \verb|NULL|. Once the symbol table is initialised, we populate the table with function mappings for all pre-defined instructions mentioned in spec. The instruction function mappings are a combination of a \verb|union| code and a parse functional pointer pointing to the appropriate parse function. 
Once the structures have been initialised, lines from the \verb|.s| file are read (all being of length \verb|511| as specified by the spec) and trailing whitespace from each line is removed and any erroneous line is ignored. Then, the line is tokenised using the \verb|tokenizer()| function in \verb|tokenizer.c|. The tokenized instruction is then appended to the linked list. Thus, the length of the link list will 1 more than the line length of the \verb|.s| file (as the head is \verb|NULL|). This linked list is passed to the parser which is part of the second pass.

\subsubsection{Second Pass: }
In the second pass



\begin{figure}[htp]
    \centering
    \includegraphics[width=15cm]{assembler_flowchart.jpg}
    \caption{Our 2-pass assembly process}
    \label{fig:assembler}
\end{figure}

\subsection{The implementation:}
% Why we chose a 2-pass approach
% Talk about the functional pointers 
% Decision choice about using parser and encoder in one function
% How the tokenizer works

Reading from the source file is required to obtain our instructions in string form. We decided to take the two-pass approach when tackling the assembler.

A notable issue we faced was that we did not know how many lines there were in each source file, so we could not create an array of strings to store these instructions. To tackle this, we used a linked list to store the instructions instead. Since we took such an approach, we could then read through the source file once, converting each instruction in a tokenized form and inserting them into the linked list. Labels are inserted into our symbol table, and are not inserted into the linked list.

\section{Extension: A Concurrent Application of Blockchain Mining in C}
\subsection{Real world usage:}
\lipsum[1-1]
\subsection{The implementation:}
\lipsum[1-1]
\subsection{Testing:}
\lipsum[1-1]
\subsubsection{Effectiveness of testing:}
\lipsum[1-1]

\section{Group Reflection}
\subsection{Effectiveness of communication and division of work:}
\lipsum[1-1]
\subsection{Future improvements:}
\lipsum[1-1]

\section{Individual Reflections}
\subsection{Ashvin:}
\lipsum[1-1]
\subsection{Kavya:}
\lipsum[1-1]
\subsection{Siddhant:}
\lipsum[1-1]
\subsection{Ye Lun:}
\lipsum[1-1]

\end{document}